\documentclass[titlepage, 12pt]{article}
\usepackage[utf8]{inputenc}
\usepackage[spanish]{babel}
\usepackage[margin=3cm]{geometry}
\usepackage{graphicx}
\usepackage{xcolor}


\begin{document}

\begin{titlepage}

	\centering
	{\scshape\LARGE Escuela Superior de Cómputo. \par}
	\vspace{1cm}
	
	\includegraphics[width=0.4\textwidth]{escom.png}\par\vspace{1cm}
	
	{\scshape\Large Web Application Development.\par}
	\vspace{1cm}
	
	{\Large\bfseries ``1er Avance Proyecto Final "\par}
	\vspace{1cm}
	
    {\Large\itshape Integrantes:  \\ Guadarrama Ascencio Juan Carlos. \\ Ornelas García Luis Ángel. \\ Sampayo Hernández Mauro. \par}
	\vspace{1cm}
	
	\vfill
	Profesor:\par
	Enrique Zarate M. en C. José Asunción.
	\vfill
	
\end{titlepage}

\tableofcontents

\newpage

\listoffigures

\newpage

\section{Introducción:}

Nuestro proyecto consiste en una aplicación web para el manejo de una base de datos de un banco. En nuestro modelo entidad relación podemos contemplar los débitos y créditos de cada cliente, además de la sucursal en dónde se encuentra registrado el mismo. Consideramos que cada cliente puede tener varias cuentas y préstamos en la misma o en diferentes sucursales. \par\vspace{0.5cm}

Nuestra aplicación contemplará el uso de servlets para dar más dinamismo a la hora de usarla.

\section{Modelo Entidad-Relación:}

    \begin{figure}[h]
    \caption{Modelo E-R.}
    \centering
    \includegraphics[width=1.0\textwidth]{Modelo E-R.png} \par\vspace{0.5cm}
    \end{figure}

\section{Diccionario de Datos:}

En esta sección mostraremos el diccionario de datos de nuestro proyecto.

    \clearpage

    \begin{figure}[h]
    \caption{Diccionario de datos: Cliente}
    \centering
    \includegraphics[width=0.8\textwidth]{DD cliente.png} \par\vspace{0.5cm}
    \end{figure}
    
    \begin{figure}[h]
    \caption{Diccionario de datos: ctacliente}
    \centering
    \includegraphics[width=0.8\textwidth]{DD ctacliente.png} \par\vspace{0.5cm}
    \end{figure}
    
    \clearpage
    
    \begin{figure}[h]
    \caption{Diccionario de datos: Prestatario}
    \centering
    \includegraphics[width=0.8\textwidth]{DD prestatario.png} \par\vspace{0.5cm}
    \end{figure}
    
    \begin{figure}[h]
    \caption{Diccionario de datos: Cuenta}
    \centering
    \includegraphics[width=0.8\textwidth]{DD cuenta.png} \par\vspace{0.5cm}
    \end{figure}
    
    \clearpage
    
    \begin{figure}[h]
    \caption{Diccionario de datos: Préstamo}
    \centering
    \includegraphics[width=0.8\textwidth]{DD prestamo.png} \par\vspace{0.5cm}
    \end{figure}
    
    \begin{figure}[h]
    \caption{Diccionario de datos: Sucursal}
    \centering
    \includegraphics[width=0.8\textwidth]{DD sucursal.png} \par\vspace{0.5cm}
    \end{figure}

\section{Referencias bibliográficas:}

[1] Data Dictionaries. 19 de septiembre del 2021, de javatpoint Sitio web: https://www.javatpoint.com/software-engineering-data-dictionaries \par\vspace{0.5cm}

[2] Manually creating a data dictionary. 19 de septiembre del 2021, de U.S. DEPARTMENT OF AGRICULTURE Sitio web: https://data.nal.usda.gov/manually-creating-data-dictionary

\end{document}